\documentclass{article}

\usepackage{amsfonts}
\usepackage{amsmath}
\usepackage{bbm}

\begin{document}

\tableofcontents

\pagebreak

\section{Introduction to Machine Learning}

\begin{itemize}
    \item Machine learning algorithms are statistical algorithms that can learn from data and generalize to unseen data, thus performing tasks without explicit instructions.
    \item In supervised methods we know the outcome Y for each input X; often, the task is to predict Y based on X.
    \item In unsupervised methods we have only the inputs X. Typical tasks are about finding structure in data.
    \begin{itemize}
        \item For example, it can be used to find "clusters".
    \end{itemize}
    \item X is often known as:
    \begin{itemize}
        \item Input
        \item Feature
        \item Predictor
        \item Covariate
        \item Independent variable
    \end{itemize}
    \item Y is often known as:
    \begin{itemize}
        \item Output
        \item Outcome
        \item Response
        \item Target variable
        \item Dependent variable
    \end{itemize}
    \item Supervised problems are often referred to as either:
    \begin{itemize}
        \item \textbf{Regression}: The outcome Y is quantitative (typically $\mathbb{R}$)s
        \item \textbf{Classification}: The outcome Y is categorical
    \end{itemize}
\end{itemize}

\pagebreak

\section{Linear Regression}

\subsection{Simple Linear Regression}
Simple linear regression models the relationship between a single predictor \(X\) and a response \(Y\) using a linear function:
\[
Y = \beta_0 + \beta_1 X + \epsilon,
\]
where:
\begin{itemize}
    \item \(\beta_0\) is the intercept,
    \item \(\beta_1\) is the slope of the line,
    \item \(\epsilon\) is the error term, which is assumed to be normally distributed with mean 0.
\end{itemize}
The coefficients \(\beta_0\) and \(\beta_1\) are estimated using the least squares criterion, which minimizes the residual sum of squares (RSS):
\[
RSS = \sum_{i=1}^{n} (y_i - \hat{y}_i)^2 = \sum_{i=1}^{n} (y_i - (\beta_0 + \beta_1 x_i))^2.
\]

\textbf{Interpretation:} 
\begin{itemize}
    \item \(\beta_1\) represents the average change in the response \(Y\) associated with a one-unit change in the predictor \(X\).
    \item For example, in predicting sales based on TV advertising spend, \(\beta_1\) would indicate how much sales increase, on average, for each additional unit of advertising expenditure on TV.
\end{itemize}

\subsection{Multiple Linear Regression}
Multiple linear regression extends the simple linear regression model to include multiple predictors:
\[
Y = \beta_0 + \beta_1 X_1 + \beta_2 X_2 + \ldots + \beta_p X_p + \epsilon.
\]
Here:
\begin{itemize}
    \item Each \(\beta_j\) represents the change in \(Y\) associated with a one-unit change in \(X_j\), holding all other predictors constant.
    \item The model can be written in matrix notation as:
    \[
    Y = X\beta + \epsilon,
    \]
    where \(Y\) is an \(n \times 1\) vector of responses, \(X\) is an \(n \times (p+1)\) matrix of predictors (including a column of ones for the intercept), \(\beta\) is a \((p+1) \times 1\) vector of coefficients, and \(\epsilon\) is an \(n \times 1\) vector of errors.
\end{itemize}

\textbf{Example: Advertising Data.} Consider a multiple regression model for predicting sales using TV, radio, and newspaper advertising budgets:
\[
\text{sales} = \beta_0 + \beta_1 \cdot \text{TV} + \beta_2 \cdot \text{radio} + \beta_3 \cdot \text{newspaper} + \epsilon.
\]
The estimated coefficients \(\beta_1, \beta_2,\) and \(\beta_3\) help to understand the individual contribution of each type of advertising while accounting for the other types.

\subsection{3.3 Other Considerations in the Regression Model}
\textbf{Qualitative Predictors:} 
Qualitative or categorical predictors can be incorporated into regression models by creating dummy variables. For example, if the predictor is a categorical variable 'region' with three levels (East, West, North), we can introduce two dummy variables:
\[
\text{region}_1 = 
\begin{cases} 
1 & \text{if East} \\
0 & \text{otherwise}
\end{cases}, \quad 
\text{region}_2 = 
\begin{cases} 
1 & \text{if West} \\
0 & \text{otherwise}
\end{cases}.
\]
Incorporating these into the regression model allows us to estimate different intercepts for each region.

\textbf{Interaction Terms:}
Interaction terms allow modeling situations where the effect of one predictor depends on another. For example, if we believe that the effect of TV advertising on sales depends on the level of radio advertising, we can include an interaction term:
\[
\text{sales} = \beta_0 + \beta_1 \cdot \text{TV} + \beta_2 \cdot \text{radio} + \beta_3 \cdot (\text{TV} \times \text{radio}) + \epsilon.
\]

\textbf{Extensions and Potential Problems:}
\begin{itemize}
    \item \textbf{Polynomial Regression:} Adds polynomial terms to the regression model to capture non-linear relationships.
    \item \textbf{Potential Issues:} Multicollinearity (high correlation among predictors), non-linearity of relationships, and outliers can affect model estimates.
\end{itemize}


\section{Bias and Variance}

Assuming $Y=f(X_0)+\varepsilon$, where:
\begin{itemize}
    \item $\mathbb{E}(\varepsilon)=0$
    \item $\varepsilon$ is independent of $X_0$
\end{itemize}
When fitting $\hat{f}$ with a dataset of $N$ observations ($X$,$Y$)
\begin{itemize}
    \item $N$ observations = \textit{training data}
    \item Algorithm for fitting $\hat{f}$ = \textit{learner}
    \item Applying learner to the training data = \textit{training}
\end{itemize}
Given training data and a chosen model framework, we want to minimize \textit{expected test MSE}: 
\begin{align*}
    \mathbb{E}[(Y_0-\hat{f}(X_0))^2]
\end{align*}
The expected test MSE at $x_0$ is:
\begin{align*}
    \mathbb{E}(Y_0-\hat{f}(x_0))=\mathbb{E}(\hat{f}(x_0)-\mathbb{E}(\hat{f}(x_0)))^2+(\mathbb{E}(\hat{f}(x_0))-f(x_0))^2+\text{Var}(\varepsilon)
\end{align*}
Where:
\begin{itemize}
    \item $\mathbb{E}(\hat{f}(x_0)-\mathbb{E}(\hat{f}(x_0)))^2$ is the Variance of $\hat{f}(x_0)$, the amount by which $\hat{f}$ would change if it were estimated with a different training data set
    \item $\mathbb{E}(\hat{f}(x_0))-f(x_0)$ is the Bias of $\hat{f}(X_0)$, the expected deviation of the model prediction from the true value - a.k.a., the inability to capture the true relationship
\end{itemize}
In the MSE, all 3 terms are non-negative, so any large term = large MSE; however, variance and bias are both \textit{reducible errors}.

\subsection{Validation sets}

If we wish to use MSE for model building - selecting features and tuning hyperparameters - we create a validation set, separate from training/testing set, for estimating the MSE.

\subsubsection{Leave-one-out crossvalidation (LOOCV)}

Create $n$ different partitions (where $n$ is amount of data points), with just one validation data point in each.
For each partition, $i$, the test MSE is $(y_i-\hat{y_i})^2$.
The LOOCV estimate for the test MSE is:
\begin{align*}
    \text{MSE} = \frac{1}{n}\sum_{i=n}^{n}(y_i-\hat{y_i})^2
\end{align*}
Downside: This is computationally expensive, requiring $n$ fits.

\subsubsection{$k$-fold crossvalidation}

Create $k$ different partitions ("folds"), in each $1/k$ of the data is for validation. Randomly shuffle data beforehand if it's ordered.

For each fold, compute $\text{MSE}_i$. Then, average:
\begin{align*}
    \frac{1}{k}\sum_{i=1}^{k}\text{MSE}_k
\end{align*}
Downside: Non-deterministic.

\section{Regression with many features}

Problem:
\begin{itemize}
    \item When features $p \approx n$, not enough data to estimate the regression curve well.
    \item When $p > n$, MLE doesn't exist.
\end{itemize}

Solutions:
\begin{itemize}
    \item Model selection
    \item Shrinkage (regularisation)
\end{itemize}

\subsection{Model selection}

As mentioned in 2.6.2, several methods:
\begin{itemize}
    \item \textbf{Best subset:} Self-explanitory, expensive
    \begin{itemize}
        \item Genetic algorithm can be used as heuristic
    \end{itemize}
    \item \textbf{Forward selection:} Start by including no/few variables
    \item \textbf{Backward selection:} Start by including all/many variables
    \item \textbf{Alternating:} Alternate between forward and backward selection
\end{itemize}

For comparing models, AIC, BIC, or test error can be used.

\subsubsection{AIC and BIC}

\begin{align*}
    \text{AIC}=-2\log\hat{L}+2p
\end{align*}
\begin{align*}
    \text{BIC}=-2\log\hat{L}+p \log n
\end{align*}
Where $p$ is the number of parameters, $n$ is the number of features, and d $\hat{L}$ is the maximised likelihood function. Smaller = better.

\subsubsection{Stepwise selection}

\textbf{Backwards:}
\begin{enumerate}
    \item Make initial model, with all/many features
    \item Iterative remove the least relevant feature (largest drop in AIC)
    \item Stop if removing a feature does not drop AIC significantly
\end{enumerate}

\textbf{Forwards:}
\begin{enumerate}
    \item Make initial model, with only intercept/few features
    \item Iterative add the most relevant feature (largest drop in AIC)
    \item Stop if adding a feature does not drop AIC significantly
\end{enumerate}

\subsubsection{Notes on automatic model selection}

\begin{itemize}
    \item One may wish to keep certain variables, even if they are not significant, if they are relevant in some other way, such as to the research question.
    \item One may wish to remove certain variables, even if they are significant, if they complicate interpretations.
\end{itemize}

\subsection{Shrinkage}

Shrinkage helps estimate coefficients $\beta$ in a constrained way, keeping them small. Doing so reduces coefficient variance and can improve the fit.

\subsubsection{Ridge regression}

Ridge regression minimizes the residual sum of squares, but adds a penalty term proportional to the squared magnitude of the coefficients:

\begin{align*}
    \hat{\beta}_\text{ridge} = \underset{\beta}{\mathrm{argmin}} \left\{ \sum_{i=1}^{n} (y_i - \beta_0 - \sum_{j=1}^{p} \beta_j x_{ij})^2 + \lambda \sum_{j=1}^{p} \beta_j^2 \right\}
\end{align*}

Where:
\begin{itemize}
    \item $\lambda \geq 0$ is a tuning parameter that controls the strength of the penalty.
    \item A larger $\lambda$ leads to more shrinkage and smaller coefficients.
    \item Ridge regression shrinks coefficients towards zero but never sets them exactly to zero.
    \item $\text{argmin}$ (short for "argument of the minimum") finds the value of $\beta$ that minimizes the function inside the curly braces. In this case, it finds the coefficients that minimize the sum of squared errors plus the penalty term.
\end{itemize}

\subsubsection{Lasso}

Lasso (Least Absolute Shrinkage and Selection Operator) regression also minimizes the residual sum of squares, but it adds a penalty term proportional to the absolute value of the coefficients:

\begin{align*}
    \hat{\beta}^\text{lasso} = \underset{\beta}{\mathrm{argmin}} \left\{ \sum_{i=1}^{n} (y_i - \beta_0 - \sum_{j=1}^{p} \beta_j x_{ij})^2 + \lambda \sum_{j=1}^{p} |\beta_j| \right\}
\end{align*}

Where:
\begin{itemize}
    \item $\lambda \geq 0$ is a tuning parameter that controls the strength of the penalty.
    \item Lasso encourages sparsity, meaning that some coefficients are set exactly to zero for sufficiently large $\lambda$.
    \item Lasso can be used for both regularization and feature selection.
\end{itemize}

\subsubsection{Lasso vs Ridge}

\textbf{TLDR}: Ridge scales everything down, while Lasso also gets rid of small enough coefficients.

\begin{itemize}
    \item \textbf{Ridge:}
    \begin{itemize}
        \item Ridge regression applies an $L_2$ penalty, which shrinks the coefficients continuously towards zero.
        \item It does not perform feature selection as it keeps all coefficients, though they may be very small.
    \end{itemize}
    
    \item \textbf{Lasso:}
    \begin{itemize}
        \item Lasso applies an $L_1$ penalty, which shrinks some coefficients to zero exactly.
        \item It performs both regularization and feature selection by setting some coefficients to zero.
    \end{itemize}
    
    \item \textbf{Choosing between them:}
    \begin{itemize}
        \item Lasso is preferred when you suspect many features are irrelevant and need automatic feature selection.
        \item Ridge is preferred when all features are potentially relevant but need regularization to prevent overfitting.
        \item Elastic net is a compromise between Ridge and Lasso, applying a combination of $L_1$ and $L_2$ penalties.
    \end{itemize}
\end{itemize}

\section{Classification with Logistic Regression}

\begin{itemize}
    \item Logistic regression seeks to:
    \begin{itemize}
        \item Model the probability of an event occurring depending on the values of the independent variables, which can be categorical or numerical
        \item Estimate the probability that and vent occurs for a randomly selected observation versus the probability that the event does not occur
        \item Predict the effect of a series of variables on a binary response (dependent) variable
        \item Classify observations by estimating the probability that an observation is in a particular category
    \end{itemize}
    \item Binary data does not have a normal distribution, which is needed for linear regression
    \item For logistic regression, we use the $\chi^2$-squared test instead of $F$-test
\end{itemize}

\subsection{Odds}

\begin{itemize}
    \item $\text{odds}=\frac{P(\text{occurring})}{P(\text{not occurring})}=\frac{p}{1-p}$
    \item $p=\frac{e^{\log(\text{odds})}}{1+e^{\log(\text{odds})}}$
    \item $\text{odds ratio}=\frac{\text{odds}_1}{\text{odds}_0}$ for two events with probabilities $p_0$ and $p_1$
    \item In logistic regression for an independent variable, the odds ratio represents how the odds change with a 1 unit increase in the variable, holding all other variables constant
    \item The odds ratio holds constant for the entire range of the variable
\end{itemize}

\subsection{Logit}

\begin{itemize}
    \item The dependent variable in logistic regression follows the Bernoulli distribution having an unknown probability $p$
    \item In logistic regression we are estimating an unknown $p$ for any given linear combination of independent variables. This estimate is $\hat{p}$
    \item We need to link our independent variables to the Bernoulli distribution; that link is called the logit
    \item Logit - the natural log of the odds ratio - maps linear combinations of variables onto the Bernoulli probability distribution with a domain from 0 to 1
    \item $\text{logit}(p)=\ln(\text{odds})=\ln(\frac{p}{1-p})=\ln(p)-\ln(1-p)$
    \item $\text{logit}$ is undefined at $p=0$ and $p=1$
    \item The logit function runs from 0 to 1 along the x-axis, not the y-axis. This we use inverse logit
    \item $\text{logit}^{-1}(\alpha)=\mu_{y|x}=\frac{1}{1+e^{-\alpha}}=\frac{e^{\alpha}}{1+e^{\alpha}}$
    \item The logit is equivalent to a linear function of the independent variables. The antilog of the logit function allows us to find the estimated regression equation
    \item $\text{logit}(p)=\ln(\frac{p}{1-p})=\beta_0+\beta_1x_1$ $\Rightarrow$ $\frac{p}{1-p}=e^{b_0+b_1 x_1}$ $\Rightarrow \dots \Rightarrow$ $\hat{p}=\frac{e^{b_0+b_1 x_1}}{1 + e^{b_0+b_1 x_1}}$, the estimated regression equation
\end{itemize}

\subsection{Estimating the Probability}

\begin{itemize}
    \item Logistic regression uses Maximum Likelihood Estimation (MLE) to estimate the parameters $\beta_0$ and $\beta_1$
    \item MLE seeks to maximize the likelihood function, which measures the probability of observing the given sample data
\end{itemize}

\subsubsection{Likelihood and Log-Likelihood}

\begin{itemize}
    \item The likelihood function $L(\beta_0, \beta_1 \mid y, x)$ is the joint probability of observing the given sample data as a function of the parameters $\beta_0$ and $\beta_1$
    \item For a binary outcome $y_i \in \{0,1\}$ with probability $p_i$ for $y_i=1$ and $(1-p_i)$ for $y_i=0$, the likelihood function for $n$ observations is:
    \[
    L(\beta_0, \beta_1) = \prod_{i=1}^{n} p_i^{y_i} (1-p_i)^{1-y_i}
    \]
    \item The log-likelihood function is the natural logarithm of the likelihood function:
    \[
    \ell(\beta_0, \beta_1) = \ln(L(\beta_0, \beta_1)) = \sum_{i=1}^{n} \left[ y_i \ln(p_i) + (1-y_i) \ln(1-p_i) \right]
    \]
    \item The log-likelihood function is easier to maximize because it converts the product of probabilities into a sum, which is mathematically simpler to handle
\end{itemize}

\subsubsection{Maximum Likelihood Estimation (MLE)}

\begin{itemize}
    \item MLE involves finding the parameter values $\beta_0$ and $\beta_1$ that maximize the log-likelihood function
    \item This is often done using iterative numerical optimization techniques, such as the Newton-Raphson method or gradient descent. There is no direct method like with gaussian (normal) models
    \item The estimated parameters $\hat{\beta}_0$ and $\hat{\beta}_1$ maximize the likelihood of observing the sample data, providing the best fit for the logistic regression model
\end{itemize}

\subsection{Multiple Logistic Regression}

\begin{itemize}
    \item Logistic regression can be extended to handle multiple predictors to model the probability of a binary outcome.
    \item The logistic regression model for multiple predictors takes the form:
    \[
    \log \left(\frac{p(X)}{1-p(X)}\right) = \beta_0 + \beta_1 X_1 + \cdots + \beta_p X_p
    \]
    \item This can be rewritten to give the estimated probability:
    \[
    p(X) = \frac{e^{\beta_0 + \beta_1 X_1 + \cdots + \beta_p X_p}}{1 + e^{\beta_0 + \beta_1 X_1 + \cdots + \beta_p X_p}}
    \]
    \item Each coefficient $\beta_i$ represents the change in the log-odds of the response per unit increase in the corresponding predictor $X_i$, holding all other predictors constant.
\end{itemize}

\subsection{Coefficients and Interpretation of Results}

\begin{itemize}
    \item The coefficients in a logistic regression model indicate the effect of each predictor on the log-odds of the outcome, holding other variables constant.
    \item A positive coefficient increases the odds of the event, while a negative coefficient decreases the odds.
    \item However, when predictors are correlated (confounding), their individual effects may appear different when considered alone versus together, potentially leading to misleading interpretations.
    \item Proper interpretation requires understanding both the individual and combined effects of predictors on the response.
\end{itemize}

\subsubsection{Making Predictions}

\begin{itemize}
    \item Using the logistic regression equation, we can predict the probability of an event occurring for a given set of predictor values.
    \item For example, for a given set of predictor values $X = (X_1, \ldots, X_p)$, the predicted probability of the event is:
    \[
    \hat{p}(X) = \frac{e^{\beta_0 + \beta_1 X_1 + \cdots + \beta_p X_p}}{1 + e^{\beta_0 + \beta_1 X_1 + \cdots + \beta_p X_p}}
    \]
\end{itemize}

\subsection{Multinomial Logistic Regression}

\begin{itemize}
    \item Multinomial logistic regression extends binary logistic regression to classify a response variable with more than two classes.
    \item For $K$ classes, we select one as the baseline (e.g., class $K$), and model the probabilities for the other classes as:
    \[
    \operatorname{Pr}(Y=k \mid X=x)=\frac{e^{\beta_{k0}+\beta_{k1} x_1 + \cdots + \beta_{kp} x_p}}{1 + \sum_{l=1}^{K-1} e^{\beta_{l0} + \beta_{l1} x_1 + \cdots + \beta_{lp} x_p}}
    \]
    for $k = 1, \ldots, K-1$, with the baseline class:
    \[
    \operatorname{Pr}(Y=K \mid X=x)=\frac{1}{1 + \sum_{l=1}^{K-1} e^{\beta_{l0} + \beta_{l1} x_1 + \cdots + \beta_{lp} x_p}}
    \]
    \item The log odds for class $k$ versus the baseline $K$ is:
    \[
    \log \left(\frac{\operatorname{Pr}(Y=k \mid X=x)}{\operatorname{Pr}(Y=K \mid X=x)}\right) = \beta_{k0} + \beta_{k1} x_1 + \cdots + \beta_{kp} x_p
    \]
    \item Coefficients are interpreted in terms of log odds ratios relative to the baseline class. A one-unit increase in $X_j$ leads to a $\beta_{kj}$ increase in the log odds of class $k$ versus the baseline.
\end{itemize}

\subsubsection{Softmax Coding}

\begin{itemize}
    \item In softmax coding, no baseline class is chosen, and probabilities for all $K$ classes are modeled symmetrically:
    \[
    \operatorname{Pr}(Y=k \mid X=x)=\frac{e^{\beta_{k0}+\beta_{k1} x_1 + \cdots + \beta_{kp} x_p}}{\sum_{l=1}^{K} e^{\beta_{l0} + \beta_{l1} x_1 + \cdots + \beta_{lp} x_p}}
    \]
    \item The log odds ratio between two classes $k$ and $k^{\prime}$ is:
    \[
    \log \left(\frac{\operatorname{Pr}(Y=k \mid X=x)}{\operatorname{Pr}(Y=k^{\prime} \mid X=x)}\right) = (\beta_{k0} - \beta_{k^{\prime}0}) + \sum_{j=1}^{p} (\beta_{kj} - \beta_{k^{\prime}j}) x_j
    \]
\end{itemize}

\section{Bayes Classifiers, K-nearest neighbours}

\begin{itemize}
  \item For classifying data points, we can use Bayes' Theorem:
  \[
  p(\mathcal{C}_k \mid X) = \frac{p(X \mid \mathcal{C}_k) p(\mathcal{C}_k)}{p(X)}
  \]
  \item We can assign a data point to the class with the highest posterior probability:
  \[
  d(x) = \arg\max_k \, p(\mathcal{C}_k \mid X)
  \]
  \item Decision regions $\mathcal{R}_k$ are separated by boundaries based on $p(\mathcal{C}_k \mid X)$.
\end{itemize}

\subsection{Loss and Loss Matrix}

\begin{itemize}
  \item The loss matrix $L_{kj}$ captures the cost of predicting class $j$ when the true class is $k$. Each element of the matrix represents how costly a mistake is. For example, a large $L_{kj}$ indicates that predicting $j$ when the true class is $k$ is a serious error.
  \item In the matrix, rows represent the true class, and columns represent the predicted class:
  \[
  L_{kj} \quad \text{(Rows: True Class, Columns: Predicted Class)}
  \]
  \item The goal is to minimize the expected loss:
  \[
  \mathbb{E}[L] = \sum_{k} \sum_{j} \int_{\mathcal{R}_j} L_{kj} p(X) p(\mathcal{C}_k \mid X) \, dX
  \]
  This formula sums over all classes $k$ and predictions $j$, calculating the total cost of assigning a point $X$ to class $j$ while the true class is $k$. The integral over region $\mathcal{R}_j$ considers all feature space where class $j$ is predicted.
  \item To minimize loss, we assign $X$ to the class $j$ that minimizes:
  \[
  \sum_{k} L_{kj} p(\mathcal{C}_k \mid X)
  \]
  Here, for each possible predicted class $j$, we sum the cost $L_{kj}$ weighted by the posterior probability $p(\mathcal{C}_k \mid X)$, selecting the class $j$ that minimizes this sum. This ensures the decision balances the cost of errors with the likelihood of each class.
\end{itemize}

\subsection{Bayes Classifier}

\begin{itemize}
  \item The Bayes classifier is a specific case of minimizing expected loss, using a 0-1 loss function, defined as:
  \[
  L(j, k) = \begin{cases} 1, & j \neq k \\ 0, & j = k \end{cases}
  \]
  This is a loss function, where all errors are equal, unlike in the loss matrix with varying costs.
  
  \item The Bayes classifier assigns a new data point $X$ to the class with the highest posterior probability:
  \[
  d(x) = \arg\max_y \, p(Y = y \mid X)
  \]
\end{itemize}

\subsection{KNN for Classification}

\begin{itemize}
  \item Find the $K$ nearest neighbors to $X_0$ and estimate class probabilities based on neighbors' classes:
  \[
  p(Y = j \mid X_0) = \frac{1}{K} \sum_{i \in \mathcal{N}_0} I(y_i = j)
  \]
  \item Assign to the class with the highest posterior probability.
\end{itemize}

\subsubsection{Choosing $K$}

\begin{itemize}
  \item Low $K$: highly flexible boundaries.
  \item High $K$: less flexible, possibly too rigid.
  \item Choose $K$ via cross-validation to optimize performance.
\end{itemize}

\section{Linear Discriminant Analysis (LDA) and Quadratic Discriminant Analysis (QDA)}

\textbf{Oversimplified TLDR:}
\begin{itemize}
    \item Both assume all classes have a Gaussian distribution
    \item \textbf{LDA:}
    \begin{itemize}
        \item Assumes equal variation (same covariance) between classes
        \item Has straight (linear) decision boundaries
        \item Works well with less data due to fewer parameters to estimate
    \end{itemize}
    \item \textbf{QDA:}
    \begin{itemize}
        \item Assumes different variation (different covariance) between classes
        \item Has curved (quadratic) decision boundaries
        \item More flexible but requires more data to avoid overfitting
    \end{itemize}
\end{itemize}

\subsection{LDA}

\begin{itemize}
  \item LDA is a generative classification model that assumes the feature vectors $\mathbf{x} \in \mathbb{R}^p$ are generated from class-specific Gaussian distributions.
  \item Key assumptions of LDA:
  \begin{itemize}
    \item The class conditional densities $p(\mathbf{x} \mid \mathcal{C}_k)$ are multivariate Gaussian distributions.
    \item All classes share a common covariance matrix $\boldsymbol{\Sigma}$.
  \end{itemize}
  \item The class conditional density function is:
  \[
  p(\mathbf{x} \mid \mathcal{C}_k) = \frac{1}{(2\pi)^{p/2} |\boldsymbol{\Sigma}|^{1/2}} \exp\left( -\frac{1}{2} (\mathbf{x} - \boldsymbol{\mu}_k)^\top \boldsymbol{\Sigma}^{-1} (\mathbf{x} - \boldsymbol{\mu}_k) \right)
  \]
  where:
  \begin{itemize}
    \item $\boldsymbol{\mu}_k$ is the mean vector of class $\mathcal{C}_k$.
    \item $|\boldsymbol{\Sigma}|$ denotes the determinant of the covariance matrix.
    \item $p$ is the number of features.
  \end{itemize}
  \item Using Bayes' theorem, the posterior probability is:
  \[
  P(\mathcal{C}_k \mid \mathbf{x}) = \frac{\pi_k \, p(\mathbf{x} \mid \mathcal{C}_k)}{p(\mathbf{x})}
  \]
  where $\pi_k = P(\mathcal{C}_k)$ is the prior probability of class $\mathcal{C}_k$.
  \item The discriminant function for LDA is linear in $\mathbf{x}$:
  \[
  \delta_k(\mathbf{x}) = \mathbf{x}^\top \boldsymbol{\Sigma}^{-1} \boldsymbol{\mu}_k - \frac{1}{2} \boldsymbol{\mu}_k^\top \boldsymbol{\Sigma}^{-1} \boldsymbol{\mu}_k + \ln \pi_k
  \]
  \item Classification rule:
  \[
  \text{Assign } \mathbf{x} \text{ to class } \mathcal{C}_k \text{ if } \delta_k(\mathbf{x}) = \max_{j} \delta_j(\mathbf{x})
  \]
  \item Decision boundaries are linear hyperplanes separating the classes.
\end{itemize}

\subsubsection{Estimation of Parameters}

\begin{itemize}
  \item \textbf{Class Means}:
  \[
  \hat{\boldsymbol{\mu}}_k = \frac{1}{n_k} \sum_{i \in \mathcal{C}_k} \mathbf{x}_i
  \]
  where $n_k$ is the number of observations in class $\mathcal{C}_k$.
  \item \textbf{Common Covariance Matrix}:
  \[
  \hat{\boldsymbol{\Sigma}} = \frac{1}{n - K} \sum_{k=1}^K \sum_{i \in \mathcal{C}_k} (\mathbf{x}_i - \hat{\boldsymbol{\mu}}_k)(\mathbf{x}_i - \hat{\boldsymbol{\mu}}_k)^\top
  \]
  where $n$ is the total number of observations and $K$ is the number of classes.
  \item \textbf{Class Priors}:
  \[
  \hat{\pi}_k = \frac{n_k}{n}
  \]
  These can also be set based on prior knowledge.
\end{itemize}

\subsection{Quadratic Discriminant Analysis (QDA)}

\begin{itemize}
  \item QDA relaxes the assumption of a common covariance matrix, allowing each class to have its own covariance matrix $\boldsymbol{\Sigma}_k$.
  \item The class conditional density function is:
  \[
  p(\mathbf{x} \mid \mathcal{C}_k) = \frac{1}{(2\pi)^{p/2} |\boldsymbol{\Sigma}_k|^{1/2}} \exp\left( -\frac{1}{2} (\mathbf{x} - \boldsymbol{\mu}_k)^\top \boldsymbol{\Sigma}_k^{-1} (\mathbf{x} - \boldsymbol{\mu}_k) \right)
  \]
  \item The discriminant function for QDA is quadratic in $\mathbf{x}$:
  \[
  \delta_k(\mathbf{x}) = -\frac{1}{2} \ln |\boldsymbol{\Sigma}_k| - \frac{1}{2} (\mathbf{x} - \boldsymbol{\mu}_k)^\top \boldsymbol{\Sigma}_k^{-1} (\mathbf{x} - \boldsymbol{\mu}_k) + \ln \pi_k
  \]
  \item Classification rule:
  \[
  \text{Assign } \mathbf{x} \text{ to class } \mathcal{C}_k \text{ if } \delta_k(\mathbf{x}) = \max_{j} \delta_j(\mathbf{x})
  \]
  \item Decision boundaries are quadratic surfaces in the feature space.
\end{itemize}

\subsubsection{Estimation of Parameters}

\begin{itemize}
  \item \textbf{Class Means}:
  \[
  \hat{\boldsymbol{\mu}}_k = \frac{1}{n_k} \sum_{i \in \mathcal{C}_k} \mathbf{x}_i
  \]
  \item \textbf{Class Covariance Matrices}:
  \[
  \hat{\boldsymbol{\Sigma}}_k = \frac{1}{n_k - 1} \sum_{i \in \mathcal{C}_k} (\mathbf{x}_i - \hat{\boldsymbol{\mu}}_k)(\mathbf{x}_i - \hat{\boldsymbol{\mu}}_k)^\top
  \]
  \item \textbf{Class Priors}:
  \[
  \hat{\pi}_k = \frac{n_k}{n}
  \]
\end{itemize}

\subsection{Comparison between LDA and QDA}

\begin{itemize}
  \item \textbf{Assumptions}:
  \begin{itemize}
    \item LDA assumes equal covariance matrices across classes.
    \item QDA allows for different covariance matrices for each class.
  \end{itemize}
  \item \textbf{Decision Boundaries}:
  \begin{itemize}
    \item LDA produces linear decision boundaries.
    \item QDA produces quadratic decision boundaries.
  \end{itemize}
  \item \textbf{Complexity}:
  \begin{itemize}
    \item LDA estimates fewer parameters, which can be advantageous with limited data.
    \item QDA requires estimation of separate covariance matrices, increasing the number of parameters.
  \end{itemize}
  \item \textbf{Bias-Variance Tradeoff}:
  \begin{itemize}
    \item LDA may have higher bias if the assumption of equal covariances is violated.
    \item QDA may have higher variance due to more parameters, potentially leading to overfitting.
  \end{itemize}
  \item \textbf{When to Use}:
  \begin{itemize}
    \item LDA is preferred when class covariances are similar and the sample size is small.
    \item QDA is suitable when class covariances differ significantly and there is sufficient data.
  \end{itemize}
\end{itemize}

\subsection{Bias-Variance Tradeoff}

\begin{itemize}
  \item \textbf{LDA}:
  \begin{itemize}
    \item Lower variance due to fewer parameters.
    \item Potentially higher bias if covariance matrices are not equal.
  \end{itemize}
  \item \textbf{QDA}:
  \begin{itemize}
    \item Lower bias as it models each class more flexibly.
    \item Higher variance due to more parameters to estimate.
  \end{itemize}
  \item The choice depends on the dataset size and whether the equal covariance assumption holds.
\end{itemize}

\subsection{Decision Rule Interpretation}

\begin{itemize}
  \item The discriminant functions $\delta_k(\mathbf{x})$ can be seen as scoring functions.
  \item The classification rule selects the class with the highest score.
  \item The decision boundaries are determined by where the discriminant functions are equal.
\end{itemize}

\subsection{Practical Considerations}

\begin{itemize}
  \item \textbf{Regularization}:
  \begin{itemize}
    \item In cases of small sample sizes, covariance estimates may be unstable.
    \item Regularization techniques can be applied to covariance matrices to improve estimates.
  \end{itemize}
  \item \textbf{Computational Efficiency}:
  \begin{itemize}
    \item LDA is computationally less intensive due to a single covariance matrix.
    \item QDA requires inversion of multiple covariance matrices.
  \end{itemize}
  \item \textbf{Assumption Checking}:
  \begin{itemize}
    \item It's important to check the assumption of normality.
    \item Techniques like plotting or statistical tests can be used to assess distributional assumptions.
  \end{itemize}
\end{itemize}

\section{Multivariate Gaussian Distribution, LDA, and QDA with Multiple Features}

\textbf{Oversimplified TLDR:}
\begin{itemize}
    \item Multivariate Gaussian distributions extend the normal distribution to multiple dimensions.
    \item \textbf{LDA with Multiple Features:}
    \begin{itemize}
        \item Assumes class conditional densities are multivariate Gaussian with a common covariance matrix.
        \item Decision boundaries are linear hyperplanes.
    \end{itemize}
    \item \textbf{QDA with Multiple Features:}
    \begin{itemize}
        \item Assumes class conditional densities are multivariate Gaussian with class-specific covariance matrices.
        \item Decision boundaries are quadratic surfaces.
    \end{itemize}
\end{itemize}

\subsection{Multivariate Data}

\begin{itemize}
    \item \textbf{Random Vectors:} If $X_1, X_2, \ldots, X_p$ are real-valued random variables, we can define a $p$-dimensional random vector $\mathbf{X}$ as:
    \[
    \mathbf{X} = \begin{bmatrix} X_1 \\ \vdots \\ X_p \end{bmatrix}
    \]
    \item \textbf{Expectation (Mean):} The expectation of $\mathbf{X}$ is:
    \[
    \mathbb{E}[\mathbf{X}] = \begin{bmatrix} \mathbb{E}[X_1] \\ \vdots \\ \mathbb{E}[X_p] \end{bmatrix}
    \]
    \item \textbf{Sample Mean:} Given $n$ observations $\{\mathbf{X}_1, \mathbf{X}_2, \ldots, \mathbf{X}_n\}$, the sample mean is:
    \[
    \hat{\boldsymbol{\mu}} = \frac{1}{n} \sum_{i=1}^n \mathbf{X}_i
    \]
    \item \textbf{Variance and Covariance Matrix:} The variance of $\mathbf{X}$ is the covariance matrix $\boldsymbol{\Sigma}$, defined as:
    \[
    \operatorname{Var}[\mathbf{X}] = \mathbb{E}\left[(\mathbf{X} - \mathbb{E}[\mathbf{X}]) (\mathbf{X} - \mathbb{E}[\mathbf{X}])^\top\right]
    \]
    where $\boldsymbol{\Sigma}$ is a $p \times p$ matrix with entries:
    \[
    \Sigma_{ij} = \operatorname{Cov}[X_i, X_j]
    \]
\end{itemize}

\subsection{The Multivariate Normal Distribution}

\begin{itemize}
    \item The $p$-dimensional \textbf{Multivariate Normal Distribution} with mean vector $\boldsymbol{\mu}$ and covariance matrix $\boldsymbol{\Sigma}$ has the probability density function:
    \[
    p(\mathbf{x}; \boldsymbol{\mu}, \boldsymbol{\Sigma}) = \frac{1}{(2\pi)^{p/2} |\boldsymbol{\Sigma}|^{1/2}} \exp\left\{ -\frac{1}{2} (\mathbf{x} - \boldsymbol{\mu})^\top \boldsymbol{\Sigma}^{-1} (\mathbf{x} - \boldsymbol{\mu}) \right\}
    \]
    \item The exponent involves the \textbf{Mahalanobis distance}:
    \[
    (\mathbf{x} - \boldsymbol{\mu})^\top \boldsymbol{\Sigma}^{-1} (\mathbf{x} - \boldsymbol{\mu}) = \|\mathbf{x} - \boldsymbol{\mu}\|_{\boldsymbol{\Sigma}}^2
    \]
    \item \textbf{Special Case:} When $\boldsymbol{\Sigma} = \mathbf{I}$ (identity matrix), the variables are independent standard normal, and the Mahalanobis distance reduces to the Euclidean distance:
    \[
    \|\mathbf{x} - \boldsymbol{\mu}\|_{\mathbf{I}}^2 = \sum_{i=1}^p (x_i - \mu_i)^2
    \]
    \item \textbf{Properties:}
    \begin{itemize}
        \item Any linear transformation or translation of a multivariate normal variable is also multivariate normal.
        \item Marginal distributions of subsets of variables are multivariate normal.
    \end{itemize}
\end{itemize}

\subsection{Linear Discriminant Analysis (LDA) with Multiple Features}

\begin{itemize}
    \item \textbf{Assumptions:}
    \begin{itemize}
        \item Each class conditional density $p(\mathbf{x} \mid \mathcal{C}_k)$ is a multivariate normal distribution with mean $\boldsymbol{\mu}_k$ and common covariance matrix $\boldsymbol{\Sigma}$:
        \[
        p(\mathbf{x} \mid \mathcal{C}_k) = \frac{1}{(2\pi)^{p/2} |\boldsymbol{\Sigma}|^{1/2}} \exp\left\{ -\frac{1}{2} (\mathbf{x} - \boldsymbol{\mu}_k)^\top \boldsymbol{\Sigma}^{-1} (\mathbf{x} - \boldsymbol{\mu}_k) \right\}
        \]
    \end{itemize}
    \item \textbf{Discriminant Function:}
    \begin{itemize}
        \item The discriminant function for class $\mathcal{C}_k$ is derived from the log of the joint probability:
        \[
        g_k(\mathbf{x}) = \ln \pi_k - \frac{1}{2} (\mathbf{x} - \boldsymbol{\mu}_k)^\top \boldsymbol{\Sigma}^{-1} (\mathbf{x} - \boldsymbol{\mu}_k)
        \]
        \item Simplifies to:
        \[
        g_k(\mathbf{x}) = \mathbf{x}^\top \boldsymbol{\Sigma}^{-1} \boldsymbol{\mu}_k - \frac{1}{2} \boldsymbol{\mu}_k^\top \boldsymbol{\Sigma}^{-1} \boldsymbol{\mu}_k + \ln \pi_k
        \]
    \end{itemize}
    \item \textbf{Classification Rule:}
    \begin{itemize}
        \item Assign $\mathbf{x}$ to the class with the highest discriminant score:
        \[
        \text{Classify } \mathbf{x} \text{ to class } \mathcal{C}_k \text{ if } g_k(\mathbf{x}) = \max_j g_j(\mathbf{x})
        \]
    \end{itemize}
    \item \textbf{Decision Boundaries:}
    \begin{itemize}
        \item The boundaries between classes are linear hyperplanes where $g_j(\mathbf{x}) = g_k(\mathbf{x})$.
    \end{itemize}
\end{itemize}

\subsection{Quadratic Discriminant Analysis (QDA) with Multiple Features}

\begin{itemize}
    \item \textbf{Assumptions:}
    \begin{itemize}
        \item Each class conditional density $p(\mathbf{x} \mid \mathcal{C}_k)$ is a multivariate normal distribution with mean $\boldsymbol{\mu}_k$ and covariance matrix $\boldsymbol{\Sigma}_k$:
        \[
        p(\mathbf{x} \mid \mathcal{C}_k) = \frac{1}{(2\pi)^{p/2} |\boldsymbol{\Sigma}_k|^{1/2}} \exp\left\{ -\frac{1}{2} (\mathbf{x} - \boldsymbol{\mu}_k)^\top \boldsymbol{\Sigma}_k^{-1} (\mathbf{x} - \boldsymbol{\mu}_k) \right\}
        \]
    \end{itemize}
    \item \textbf{Discriminant Function:}
    \begin{itemize}
        \item The discriminant function for class $\mathcal{C}_k$ is:
        \[
        g_k(\mathbf{x}) = \ln \pi_k - \frac{1}{2} \ln |\boldsymbol{\Sigma}_k| - \frac{1}{2} (\mathbf{x} - \boldsymbol{\mu}_k)^\top \boldsymbol{\Sigma}_k^{-1} (\mathbf{x} - \boldsymbol{\mu}_k)
        \]
    \end{itemize}
    \item \textbf{Classification Rule:}
    \begin{itemize}
        \item Assign $\mathbf{x}$ to the class with the highest $g_k(\mathbf{x})$.
    \end{itemize}
    \item \textbf{Decision Boundaries:}
    \begin{itemize}
        \item The boundaries between classes are quadratic surfaces defined by $g_j(\mathbf{x}) = g_k(\mathbf{x})$.
    \end{itemize}
\end{itemize}

\subsection{Constructing Decision Boundaries}

\begin{itemize}
    \item \textbf{LDA Decision Boundary:}
    \begin{itemize}
        \item Between classes $\mathcal{C}_j$ and $\mathcal{C}_k$, the decision boundary is where:
        \[
        g_j(\mathbf{x}) = g_k(\mathbf{x})
        \]
        \item Simplifies to a linear equation in $\mathbf{x}$.
    \end{itemize}
    \item \textbf{Effect of Class Priors:}
    \begin{itemize}
        \item Increasing the prior probability $\pi_k$ shifts the decision boundary away from $\boldsymbol{\mu}_k$.
        \item The term $\ln \pi_k$ influences the discriminant function.
    \end{itemize}
    \item \textbf{QDA Decision Boundary:}
    \begin{itemize}
        \item Boundaries are more complex due to class-specific covariance matrices.
        \item The equation $g_j(\mathbf{x}) = g_k(\mathbf{x})$ results in quadratic terms.
    \end{itemize}
\end{itemize}

\subsection{Issues with LDA and QDA}

\begin{itemize}
    \item \textbf{Number of Parameters:}
    \begin{itemize}
        \item Covariance matrix $\boldsymbol{\Sigma}$ has $\frac{p(p+1)}{2}$ parameters.
        \item LDA estimates one covariance matrix; QDA estimates one per class.
    \end{itemize}
    \item \textbf{Bias-Variance Trade-Off:}
    \begin{itemize}
        \item LDA has lower variance but higher bias if the assumption of equal covariances is incorrect.
        \item QDA has lower bias but higher variance due to more parameters.
    \end{itemize}
    \item \textbf{Naive Bayes Classifier:}
    \begin{itemize}
        \item Assumes independence among features, leading to diagonal covariance matrices.
        \item Simplifies computation but may not capture feature correlations.
    \end{itemize}
\end{itemize}

\subsection{Case Studies}

\subsubsection{Case 1: Equal Covariance Matrices (No Correlations)}

\begin{itemize}
    \item Covariance matrices are equal and features are uncorrelated.
    \item LDA performs well due to matching assumptions.
    \item Logistic regression and Naive Bayes provide similar performance.
\end{itemize}

\subsubsection{Case 2: Unequal Covariance Matrices (No Correlations)}

\begin{itemize}
    \item Covariance matrices differ but features remain uncorrelated.
    \item QDA outperforms LDA as it models class-specific covariances.
    \item Logistic regression may require additional features (e.g., interaction terms) to capture the non-linearity.
\end{itemize}

\subsubsection{Case 3: Unequal Covariance Matrices (With Correlations)}

\begin{itemize}
    \item Covariance matrices differ and features are correlated.
    \item QDA and logistic regression with interaction terms perform best.
    \item Naive Bayes performs poorly due to the independence assumption.
\end{itemize}

\subsection{Test Error Comparison}

\begin{center}
\begin{tabular}{lccccc}
\hline
 & LDA & QDA & Naive Bayes & Logistic Regression & Logistic Regression (Quadratic) \\
\hline
Case 1 & 1.20\% & 1.30\% & 1.33\% & 1.40\% & 1.33\% \\
Case 2 & 8.00\% & 7.27\% & 7.43\% & 7.60\% & 7.57\% \\
Case 3 & 2.27\% & 0.63\% & 2.20\% & 0.87\% & 0.60\% \\
\hline
\end{tabular}
\end{center}

\subsection{LDA vs Logistic Regression}

\begin{itemize}
    \item \textbf{Relationship:}
    \begin{itemize}
        \item Both produce linear decision boundaries under certain conditions.
        \item LDA assumes normality and equal covariances; logistic regression does not.
    \end{itemize}
    \item \textbf{Log-Posterior Odds:}
    \begin{itemize}
        \item In logistic regression, the log-odds are modeled directly:
        \[
        \log\left( \frac{P(\mathcal{C}_k \mid \mathbf{x})}{P(\mathcal{C}_K \mid \mathbf{x})} \right) = \boldsymbol{\beta}_k^\top \mathbf{x} + \beta_{0k}
        \]
        \item LDA implies a linear model for log-odds when the normality assumptions hold.
    \end{itemize}
    \item \textbf{Advantages of LDA:}
    \begin{itemize}
        \item More efficient if the normality and equal covariance assumptions are valid.
        \item Naturally incorporates class priors.
    \end{itemize}
    \item \textbf{Advantages of Logistic Regression:}
    \begin{itemize}
        \item Flexible in including various types of predictors and transformations.
        \item Does not rely on distributional assumptions.
    \end{itemize}
\end{itemize}

\subsection{Practical Considerations}

\begin{itemize}
    \item \textbf{Regularization:}
    \begin{itemize}
        \item In high dimensions, covariance estimates may be unstable.
        \item Regularization techniques can improve estimates (e.g., shrinkage methods).
    \end{itemize}
    \item \textbf{Computational Efficiency:}
    \begin{itemize}
        \item LDA is computationally less intensive than QDA.
        \item QDA requires inversion of multiple covariance matrices.
    \end{itemize}
    \item \textbf{Assumption Checking:}
    \begin{itemize}
        \item Check normality of features using plots or tests.
        \item Verify the homogeneity of covariances for LDA.
    \end{itemize}
    \item \textbf{Feature Engineering:}
    \begin{itemize}
        \item Including interaction terms or polynomial features can improve logistic regression.
        \item Dimensionality reduction techniques (e.g., PCA) may be beneficial.
    \end{itemize}
\end{itemize}

\section{Performance Metrics in Classification}

\textbf{Oversimplified TLDR:}
\begin{itemize}
    \item Classification performance is assessed using various metrics beyond simple accuracy.
    \item \textbf{Confusion Matrix:} Summarizes the correct and incorrect predictions for each class.
    \item \textbf{Precision:} Measures the proportion of positive identifications that are correct.
    \item \textbf{Recall:} Measures the proportion of actual positives correctly identified.
    \item \textbf{F1-Score:} Harmonic mean of precision and recall; balances both metrics.
    \item \textbf{ROC Curve:} Plots true positive rate against false positive rate at various thresholds.
    \item \textbf{Adjusting Decision Threshold:} Changes the balance between precision and recall.
\end{itemize}

\subsection{Confusion Matrix}

\begin{itemize}
    \item A confusion matrix is a table that describes the performance of a classification model.
    \item For binary classification, it is a $2 \times 2$ matrix:
    \[
    \begin{tabular}{|c|c|c|}
    \hline
    & \textbf{Predicted Positive} & \textbf{Predicted Negative} \\
    \hline
    \textbf{Actual Positive} & \text{True Positive (TP)} & \text{False Negative (FN)} \\
    \hline
    \textbf{Actual Negative} & \text{False Positive (FP)} & \text{True Negative (TN)} \\
    \hline
    \end{tabular}
    \]
    \item Definitions:
    \begin{itemize}
        \item \textbf{True Positive (TP):} Correctly predicted positive instances.
        \item \textbf{False Positive (FP):} Incorrectly predicted positive instances.
        \item \textbf{False Negative (FN):} Incorrectly predicted negative instances.
        \item \textbf{True Negative (TN):} Correctly predicted negative instances.
    \end{itemize}
    \item The confusion matrix can be extended to multi-class classification problems.
\end{itemize}

\subsubsection{Stock Price Prediction Example}

\begin{itemize}
    \item Task: Predict stock price movement (\textit{Increase}, \textit{Decrease}, \textit{Unchanged}).
    \item Sample confusion matrix:
    \[
    \begin{tabular}{|c|c|c|c|c|}
    \hline
    & \multicolumn{3}{c|}{\textbf{Predicted Class}} & \textbf{Total} \\
    \hline
    \textbf{Actual Class} & \text{Increase} & \text{Decrease} & \text{Unchanged} & \\
    \hline
    \text{Increase} & 9 & 3 & 28 & 40 \\
    \hline
    \text{Decrease} & 1 & 5 & 34 & 40 \\
    \hline
    \text{Unchanged} & 4 & 3 & 33 & 40 \\
    \hline
    \textbf{Total} & 14 & 11 & 95 & 120 \\
    \hline
    \end{tabular}
    \]
    \item \textbf{Accuracy}:
    \[
    \text{Accuracy} = \frac{9 + 5 + 33}{120} = \frac{47}{120} \approx 0.392
    \]
    \item \textbf{Misclassification Rate}:
    \[
    \text{Misclassification Rate} = 1 - \text{Accuracy} \approx 0.608
    \]
\end{itemize}

\subsection{Accuracy and Misclassification Rate}

\begin{itemize}
    \item \textbf{Accuracy} measures the proportion of correctly classified instances:
    \[
    \text{Accuracy} = \frac{TP + TN}{TP + FP + FN + TN}
    \]
    \item \textbf{Misclassification Rate} is the proportion of incorrectly classified instances:
    \[
    \text{Misclassification Rate} = \frac{FP + FN}{TP + FP + FN + TN}
    \]
    \item High accuracy can be misleading in imbalanced datasets.
\end{itemize}

\subsection{Precision and Recall}

\begin{itemize}
    \item \textbf{Precision} (Positive Predictive Value):
    \[
    \text{Precision} = \frac{TP}{TP + FP}
    \]
    \item \textbf{Recall} (Sensitivity or True Positive Rate):
    \[
    \text{Recall} = \frac{TP}{TP + FN}
    \]
    \item These metrics help understand the model's ability to correctly predict positive instances.
\end{itemize}

\subsubsection{Calculating Precision and Recall for "Increase" Class}

\begin{itemize}
    \item For the \textit{Increase} class:
    \begin{itemize}
        \item \textbf{True Positives (TP)}: 9
        \item \textbf{False Positives (FP)}: Predictions of \textit{Increase} when actual is not \textit{Increase}:
        \[
        \text{FP} = \text{Predicted Increase Total} - \text{TP} = 14 - 9 = 5
        \]
        \item \textbf{False Negatives (FN)}: Actual \textit{Increase} instances predicted as other classes:
        \[
        \text{FN} = \text{Actual Increase Total} - \text{TP} = 40 - 9 = 31
        \]
    \end{itemize}
    \item \textbf{Precision for Increase}:
    \[
    \text{Precision}_{\text{Increase}} = \frac{9}{9 + 5} = \frac{9}{14} \approx 0.643
    \]
    \item \textbf{Recall for Increase}:
    \[
    \text{Recall}_{\text{Increase}} = \frac{9}{9 + 31} = \frac{9}{40} = 0.225
    \]
    \item \textbf{Interpretation}:
    \begin{itemize}
        \item Precision of 0.643 means that 64.3\% of predicted \textit{Increase} instances are correct.
        \item Recall of 0.225 indicates that 22.5\% of actual \textit{Increase} instances are correctly identified.
    \end{itemize}
\end{itemize}

\subsubsection{When to Use Precision vs. Recall}

\begin{itemize}
    \item \textbf{High Precision} is important when false positives are costly.
    \begin{itemize}
        \item Example: Diagnosing a rare disease where treatment is expensive or has side effects.
    \end{itemize}
    \item \textbf{High Recall} is crucial when missing positive cases is costly.
    \begin{itemize}
        \item Example: Screening for cancer where early detection is critical.
    \end{itemize}
\end{itemize}

\subsection{F1-Score}

\begin{itemize}
    \item The F1-score combines precision and recall:
    \[
    F1 = 2 \times \frac{\text{Precision} \times \text{Recall}}{\text{Precision} + \text{Recall}}
    \]
    \item For the \textit{Increase} class:
    \[
    F1_{\text{Increase}} = 2 \times \frac{0.643 \times 0.225}{0.643 + 0.225} \approx 0.333
    \]
    \item The F1-score is low when either precision or recall is low.
\end{itemize}

\subsection{Receiver Operating Characteristic (ROC) Curve}

\begin{itemize}
    \item The ROC curve plots the \textbf{True Positive Rate (TPR)} against the \textbf{False Positive Rate (FPR)} at various thresholds.
    \item \textbf{True Positive Rate (TPR)}:
    \[
    \text{TPR} = \text{Recall} = \frac{TP}{TP + FN}
    \]
    \item \textbf{False Positive Rate (FPR)}:
    \[
    \text{FPR} = \frac{FP}{FP + TN}
    \]
    \item The area under the ROC curve (AUC) indicates the classifier's ability to distinguish between classes.
    \item A higher AUC represents better performance.
\end{itemize}

\subsubsection{Interpreting the ROC Curve}

\begin{itemize}
    \item The ROC curve helps in selecting an optimal threshold by visualizing the trade-off between TPR and FPR.
    \item It is particularly useful in imbalanced datasets.
\end{itemize}

\subsection{Adjusting the Decision Threshold}

\begin{itemize}
    \item Classification models often output probabilities or scores.
    \item The decision threshold determines the point above which instances are classified as positive.
    \item Adjusting the threshold affects precision and recall:
    \begin{itemize}
        \item Increasing the threshold generally increases precision but decreases recall.
        \item Decreasing the threshold generally increases recall but decreases precision.
    \end{itemize}
    \item \textbf{Example}:
    \begin{itemize}
        \item If the cost of false negatives is high, set a lower threshold to increase recall.
        \item If the cost of false positives is high, set a higher threshold to increase precision.
    \end{itemize}
\end{itemize}

\subsubsection{Impact on Confusion Matrix}

\begin{itemize}
    \item As the threshold changes, the numbers in the confusion matrix change accordingly.
    \item This can be visualized by plotting precision and recall against different thresholds.
\end{itemize}

\subsection{Practical Considerations}

\begin{itemize}
    \item \textbf{Choice of Metric}:
    \begin{itemize}
        \item Select metrics that align with the specific goals and context of the task.
        \item Be cautious of relying solely on accuracy.
    \end{itemize}
    \item \textbf{Model Evaluation}:
    \begin{itemize}
        \item Use cross-validation to assess model performance.
        \item Compare models using relevant metrics.
    \end{itemize}
    \item \textbf{Model Interpretability}:
    \begin{itemize}
        \item Consider the trade-off between model complexity and interpretability.
        \item Simpler models may be preferred if they offer similar performance.
    \end{itemize}
\end{itemize}

\subsection{Comparing to Random Classifiers}

\begin{itemize}
    \item A model should perform better than random guessing.
    \item To assess this, compare the model's performance to that of a classifier trained on shuffled labels.
    \item If the model significantly outperforms the random baseline, it suggests that it has learned meaningful patterns.
\end{itemize}

\subsection{Final Thoughts}

\begin{itemize}
    \item Avoid focusing solely on improving numerical metrics without considering practical implications.
    \item Ensure the model is sensible, interpretable, and aligns with domain knowledge.
    \item Consider the costs associated with different types of errors when selecting thresholds and metrics.
\end{itemize}

\section{Ensemble Methods: Bagging, Random Forests, and Boosting}

\textbf{Oversimplified TLDR:}
\begin{itemize}
    \item \textbf{Ensemble Methods} combine multiple models to improve performance and robustness.
    \item \textbf{Bagging (Bootstrap Aggregation):}
    \begin{itemize}
        \item Reduces variance by training multiple models on different bootstrapped subsets of data.
        \item Aggregates predictions via averaging (regression) or majority voting (classification).
        \item Particularly effective for high-variance, low-bias models like decision trees.
    \end{itemize}
    \item \textbf{Random Forests:}
    \begin{itemize}
        \item An extension of bagging using decision trees.
        \item Introduces additional randomness by selecting random subsets of features at each split.
        \item Further reduces variance and combats overfitting.
    \end{itemize}
    \item \textbf{Boosting:}
    \begin{itemize}
        \item Reduces bias by sequentially training models that focus on correcting previous errors.
        \item Models are weighted based on their performance.
        \item Aggregates predictions using weighted voting or summation.
    \end{itemize}
\end{itemize}

\subsection{Introduction to Ensemble Methods}

\begin{itemize}
    \item \textbf{Ensemble Methods} aim to improve the performance of machine learning models by combining multiple learners.
    \item The key idea is that a group of weak learners can come together to form a strong learner.
    \item Ensemble methods can be categorized based on whether they primarily reduce variance or bias.
\end{itemize}

\subsection{Reducing Variance with Ensemble Methods}

\begin{itemize}
    \item Variance refers to the model's sensitivity to fluctuations in the training data.
    \item High-variance models, like decision trees, can benefit from ensemble methods that reduce variance.
    \item \textbf{Key Strategy:} Combine multiple independent models to average out errors.
\end{itemize}

\subsubsection{Need for Diversity}

\begin{itemize}
    \item For an ensemble to effectively reduce variance, the base learners should be diverse and make uncorrelated errors.
    \item Diversity can be introduced by:
    \begin{itemize}
        \item \textbf{Using Different Predictors:} Training different types of classifiers on the same data.
        \item \textbf{Varying Training Data:} Training the same classifier type on different subsets of the data.
    \end{itemize}
\end{itemize}

\subsection{Bagging (Bootstrap Aggregation)}

\begin{itemize}
    \item \textbf{Bagging} stands for \textbf{Bootstrap Aggregating}.
    \item It reduces variance by creating multiple versions of a predictor and using them to get an aggregated predictor.
\end{itemize}

\subsubsection{Training Procedure}

\begin{enumerate}
    \item Generate $m$ bootstrapped datasets $\{\mathcal{D}_1, \mathcal{D}_2, \dots, \mathcal{D}_m\}$ from the original training data $\mathcal{D}$.
    \begin{itemize}
        \item Each bootstrapped dataset is created by sampling $n$ instances from $\mathcal{D}$ with replacement.
    \end{itemize}
    \item Train a separate base learner $h_i$ on each bootstrapped dataset $\mathcal{D}_i$.
\end{enumerate}

\subsubsection{Prediction Procedure}

\begin{itemize}
    \item For regression:
    \[
    \hat{f}_{\text{bag}}(\mathbf{x}) = \frac{1}{m} \sum_{i=1}^{m} h_i(\mathbf{x})
    \]
    \item For classification:
    \[
    \hat{y}_{\text{bag}} = \text{MajorityVote}\left\{ h_1(\mathbf{x}), h_2(\mathbf{x}), \dots, h_m(\mathbf{x}) \right\}
    \]
\end{itemize}

\subsubsection{Out-of-Bag (OOB) Error Estimation}

\begin{itemize}
    \item Approximately $37\%$ of the original data are not included in each bootstrapped dataset (since $(1 - \frac{1}{n})^n \approx \frac{1}{e} \approx 0.37$).
    \item These \textbf{Out-of-Bag} samples can be used to estimate the prediction error without the need for a separate validation set.
    \item \textbf{Procedure:}
    \begin{enumerate}
        \item For each instance, aggregate the predictions from the models where that instance was not included in the training data.
        \item Compute the OOB error as the average error across all instances.
    \end{enumerate}
\end{itemize}

\subsubsection{Advantages of Bagging}

\begin{itemize}
    \item Reduces variance without increasing bias significantly.
    \item Easy to implement and parallelize.
    \item Provides an unbiased estimation of the generalization error via OOB error.
    \item Improves stability and reduces overfitting for high-variance models.
\end{itemize}

\subsection{Random Forests}

\begin{itemize}
    \item \textbf{Random Forests} are an extension of bagging, specifically applied to decision trees.
    \item They introduce additional randomness in the model-building process.
\end{itemize}

\subsubsection{Training Procedure}

\begin{enumerate}
    \item Generate $m$ bootstrapped datasets from the original data.
    \item For each node in a decision tree:
    \begin{itemize}
        \item Randomly select $k$ features from the total $p$ features.
        \item Determine the best split among the selected $k$ features.
    \end{itemize}
    \item Train a full decision tree on each bootstrapped dataset using the modified splitting criterion.
\end{enumerate}

\subsubsection{Hyperparameters}

\begin{itemize}
    \item Number of trees $m$.
    \item Number of features to consider at each split $k$.
    \begin{itemize}
        \item For classification: $k = \sqrt{p}$.
        \item For regression: $k = \frac{p}{3}$.
    \end{itemize}
\end{itemize}

\subsubsection{Advantages of Random Forests}

\begin{itemize}
    \item Further reduces variance compared to bagging.
    \item Handles high-dimensional data well.
    \item Provides estimates of feature importance.
    \item Less prone to overfitting due to feature randomness.
\end{itemize}

\subsection{Boosting}

\begin{itemize}
    \item \textbf{Boosting} aims to reduce bias by combining weak learners sequentially.
    \item Each new model focuses on the errors of the previous models.
    \item Particularly useful when the base learner is simple and exhibits high bias.
\end{itemize}

\subsubsection{AdaBoost (Adaptive Boosting)}

\begin{itemize}
    \item Commonly used for classification problems.
    \item Adjusts the weights of training instances based on classification errors.
\end{itemize}

\paragraph{Training Procedure}

\begin{enumerate}
    \item Initialize weights $w_i^{(1)} = \frac{1}{n}$ for $i = 1, 2, \dots, n$.
    \item For $c = 1$ to $M$ (number of iterations):
    \begin{enumerate}
        \item Train a weak learner $h_c(\mathbf{x})$ using weights $w_i^{(c)}$.
        \item Compute the weighted error:
        \[
        \epsilon_c = \frac{\sum_{i=1}^n w_i^{(c)} \mathbb{I}\left( h_c(\mathbf{x}_i) \neq y_i \right)}{\sum_{i=1}^n w_i^{(c)}}
        \]
        \item Compute the model's weight (amount of say):
        \[
        \alpha_c = \frac{1}{2} \ln\left( \frac{1 - \epsilon_c}{\epsilon_c} \right)
        \]
        \item Update the instance weights:
        \[
        w_i^{(c+1)} = w_i^{(c)} \times \begin{cases}
            e^{-\alpha_c}, & \text{if } h_c(\mathbf{x}_i) = y_i \\
            e^{\alpha_c}, & \text{if } h_c(\mathbf{x}_i) \neq y_i \\
        \end{cases}
        \]
        \item Normalize the weights so that $\sum_{i=1}^n w_i^{(c+1)} = 1$.
    \end{enumerate}
\end{enumerate}

\paragraph{Prediction Procedure}

\begin{itemize}
    \item The final prediction is a weighted majority vote:
    \[
    H(\mathbf{x}) = \text{sign}\left( \sum_{c=1}^M \alpha_c h_c(\mathbf{x}) \right)
    \]
\end{itemize}

\subsubsection{Incorporating Weights into Training}

\begin{itemize}
    \item Weights can be incorporated by:
    \begin{itemize}
        \item Sampling training instances according to their weights.
        \item Modifying the loss function to account for instance weights.
        \item Adjusting the impurity measures in decision trees using weighted counts.
    \end{itemize}
\end{itemize}

\subsubsection{Gradient Boosting}

\begin{itemize}
    \item A generalization of boosting for both regression and classification.
    \item Models are trained to predict the residuals (errors) of the previous models.
\end{itemize}

\paragraph{Training Procedure}

\begin{enumerate}
    \item Initialize the model with a constant value:
    \[
    F_0(\mathbf{x}) = \arg\min_{\gamma} \sum_{i=1}^n L(y_i, \gamma)
    \]
    \item For $m = 1$ to $M$ (number of iterations):
    \begin{enumerate}
        \item Compute the pseudo-residuals:
        \[
        r_i^{(m)} = -\left[ \frac{\partial L(y_i, F(\mathbf{x}_i))}{\partial F(\mathbf{x}_i)} \right]_{F(\mathbf{x}) = F_{m-1}(\mathbf{x})}
        \]
        \item Fit a base learner $h_m(\mathbf{x})$ to the pseudo-residuals.
        \item Update the model:
        \[
        F_m(\mathbf{x}) = F_{m-1}(\mathbf{x}) + \nu h_m(\mathbf{x})
        \]
        \item $\nu$ is the learning rate (typically between $0.01$ and $0.1$).
    \end{enumerate}
\end{enumerate}

\paragraph{Advantages of Gradient Boosting}

\begin{itemize}
    \item Can optimize arbitrary differentiable loss functions.
    \item Effective for both regression and classification.
    \item Provides state-of-the-art results on many problems.
\end{itemize}

\subsection{Bagging vs. Boosting}

\begin{itemize}
    \item \textbf{Bagging:}
    \begin{itemize}
        \item Reduces variance.
        \item Models are trained independently (can be parallelized).
        \item Effective for high-variance, low-bias models.
    \end{itemize}
    \item \textbf{Boosting:}
    \begin{itemize}
        \item Reduces bias.
        \item Models are trained sequentially (cannot be fully parallelized).
        \item Effective for low-variance, high-bias models.
        \item Can be prone to overfitting, especially with noisy data.
    \end{itemize}
\end{itemize}

\subsection{Stacking}

\begin{itemize}
    \item \textbf{Stacking} involves training a meta-learner to combine the predictions of several base learners.
    \item Base learners are trained on the original dataset.
    \item The meta-learner is trained on the outputs of the base learners.
\end{itemize}

\subsubsection{Procedure}

\begin{enumerate}
    \item Split the training data into two sets.
    \item Train base learners on the first set.
    \item Generate predictions for the second set using the trained base learners.
    \item Train a meta-learner on the predictions from the base learners.
\end{enumerate}

\subsubsection{Advantages of Stacking}

\begin{itemize}
    \item Can model complex relationships between base learners.
    \item Often achieves better performance than individual models.
\end{itemize}

\subsection{Practical Considerations}

\begin{itemize}
    \item \textbf{Choice of Ensemble Method:}
    \begin{itemize}
        \item Depends on the problem, data size, and computational resources.
        \item Bagging is suitable when variance is high.
        \item Boosting is suitable when bias is high.
    \end{itemize}
    \item \textbf{Computational Efficiency:}
    \begin{itemize}
        \item Bagging and Random Forests can be parallelized.
        \item Boosting requires sequential training.
    \end{itemize}
    \item \textbf{Overfitting:}
    \begin{itemize}
        \item Boosting can overfit on noisy datasets.
        \item Regularization techniques and early stopping can mitigate overfitting.
    \end{itemize}
\end{itemize}

\section{Support Vector Machines (SVM)}

\textbf{Oversimplified TLDR:}
\begin{itemize}
    \item SVM aims to find the optimal hyperplane that separates classes by maximizing the margin between them.
    \item \textbf{Hard-Margin SVM:}
    \begin{itemize}
        \item Assumes data is perfectly linearly separable.
        \item Maximizes the minimum distance (margin) to the closest data points from each class.
        \item Sensitive to outliers and cannot handle misclassifications.
    \end{itemize}
    \item \textbf{Soft-Margin SVM:}
    \begin{itemize}
        \item Allows some misclassifications to achieve better generalization.
        \item Introduces slack variables and a regularization parameter $C$ to control the trade-off between maximizing margin and minimizing classification error.
        \item More robust to noisy data and outliers.
    \end{itemize}
    \item \textbf{Kernel Trick:}
    \begin{itemize}
        \item Projects data into higher-dimensional space to handle non-linear separations.
        \item Uses kernel functions to compute inner products in transformed space without explicit mapping.
        \item Common kernels: Linear, Polynomial, Radial Basis Function (RBF), Sigmoid.
    \end{itemize}
\end{itemize}

\subsection{Introduction}

\begin{itemize}
    \item Support Vector Machines (SVM) are supervised learning models used for classification and regression tasks.
    \item SVM aims to find a hyperplane that best separates classes in the feature space.
    \item The optimal hyperplane is the one that maximizes the margin between the classes.
    \item Key concepts:
    \begin{itemize}
        \item \textbf{Margin:} The distance between the hyperplane and the nearest data points from each class.
        \item \textbf{Support Vectors:} The data points that lie closest to the hyperplane and influence its position.
    \end{itemize}
\end{itemize}

\subsection{Hard-Margin SVM}

\subsubsection{Assumptions and Definitions}

\begin{itemize}
    \item Applicable when data is linearly separable.
    \item Goal: Find a hyperplane that separates the classes with maximum margin.
    \item Linear classifier is defined as:
    \[
    f(\mathbf{x}) = \mathbf{w}^\top \mathbf{x} + b
    \]
    where:
    \begin{itemize}
        \item $\mathbf{w}$ is the weight vector (normal to the hyperplane).
        \item $b$ is the bias term.
    \end{itemize}
    \item Decision rule:
    \[
    \text{Predict class } \begin{cases}
    +1 & \text{if } f(\mathbf{x}) \geq 0 \\
    -1 & \text{if } f(\mathbf{x}) < 0
    \end{cases}
    \]
\end{itemize}

\subsubsection{Optimization Problem}

\begin{itemize}
    \item The margin is defined as $\dfrac{2}{\|\mathbf{w}\|}$.
    \item To maximize the margin, we minimize $\|\mathbf{w}\|$ subject to correct classification.
    \item Constraints for all training samples $(\mathbf{x}_i, y_i)$:
    \[
    y_i (\mathbf{w}^\top \mathbf{x}_i + b) \geq 1
    \]
    \item The optimization problem becomes:
    \[
    \begin{aligned}
    & \underset{\mathbf{w}, b}{\text{minimize}} && \frac{1}{2} \|\mathbf{w}\|^2 \\
    & \text{subject to} && y_i (\mathbf{w}^\top \mathbf{x}_i + b) \geq 1, \quad \forall i
    \end{aligned}
    \]
\end{itemize}

\subsubsection{Support Vectors}

\begin{itemize}
    \item Support vectors are the data points that lie on the margin boundaries (i.e., satisfy $y_i (\mathbf{w}^\top \mathbf{x}_i + b) = 1$).
    \item These points directly influence the position and orientation of the hyperplane.
    \item Data points further away from the hyperplane do not affect the solution.
\end{itemize}

\subsection{Soft-Margin SVM}

\subsubsection{Motivation}

\begin{itemize}
    \item Real-world data is often not perfectly linearly separable.
    \item Hard-margin SVM cannot handle misclassifications or noisy data.
    \item Soft-margin SVM introduces slack variables to allow some misclassifications.
\end{itemize}

\subsubsection{Mathematical Formulation}

\begin{itemize}
    \item Introduce slack variables $\xi_i \geq 0$ to allow violations of the margin constraints.
    \item Modified constraints:
    \[
    y_i (\mathbf{w}^\top \mathbf{x}_i + b) \geq 1 - \xi_i
    \]
    \item The optimization problem becomes:
    \[
    \begin{aligned}
    & \underset{\mathbf{w}, b, \boldsymbol{\xi}}{\text{minimize}} && \frac{1}{2} \|\mathbf{w}\|^2 + C \sum_{i=1}^n \xi_i \\
    & \text{subject to} && y_i (\mathbf{w}^\top \mathbf{x}_i + b) \geq 1 - \xi_i, \quad \forall i \\
    &&& \xi_i \geq 0, \quad \forall i
    \end{aligned}
    \]
    where:
    \begin{itemize}
        \item $C > 0$ is the regularization parameter controlling the trade-off between margin maximization and classification error.
    \end{itemize}
\end{itemize}

\subsubsection{Regularization Parameter $C$}

\begin{itemize}
    \item A large $C$ assigns a higher penalty to misclassifications, leading to a smaller margin.
    \item A small $C$ allows more misclassifications, leading to a larger margin.
    \item $C$ controls the bias-variance trade-off:
    \begin{itemize}
        \item High $C$: Low bias, high variance.
        \item Low $C$: High bias, low variance.
    \end{itemize}
\end{itemize}

\subsection{Non-linear SVM and the Kernel Trick}

\subsubsection{Feature Transformation}

\begin{itemize}
    \item Linear SVMs cannot handle non-linearly separable data in the original feature space.
    \item Idea: Map data to a higher-dimensional feature space where it becomes linearly separable.
    \item Mapping function $\phi: \mathbb{R}^n \rightarrow \mathbb{R}^N$, where $N > n$.
\end{itemize}

\subsubsection{Kernel Trick}

\begin{itemize}
    \item Computing $\phi(\mathbf{x})$ explicitly can be computationally expensive or infeasible.
    \item The kernel trick allows computation of inner products in the high-dimensional space without explicit mapping.
    \item Kernel function $K(\mathbf{x}_i, \mathbf{x}_j) = \phi(\mathbf{x}_i)^\top \phi(\mathbf{x}_j)$.
    \item The optimization problem depends only on inner products of the form $\mathbf{x}_i^\top \mathbf{x}_j$, which can be replaced with $K(\mathbf{x}_i, \mathbf{x}_j)$.
\end{itemize}

\subsubsection{Common Kernel Functions}

\begin{itemize}
    \item \textbf{Linear Kernel:}
    \[
    K(\mathbf{x}_i, \mathbf{x}_j) = \mathbf{x}_i^\top \mathbf{x}_j
    \]
    \item \textbf{Polynomial Kernel:}
    \[
    K(\mathbf{x}_i, \mathbf{x}_j) = (\gamma \mathbf{x}_i^\top \mathbf{x}_j + r)^d
    \]
    where $\gamma$, $r$, and $d$ are kernel parameters.
    \item \textbf{Radial Basis Function (RBF) Kernel:}
    \[
    K(\mathbf{x}_i, \mathbf{x}_j) = \exp\left( -\gamma \| \mathbf{x}_i - \mathbf{x}_j \|^2 \right)
    \]
    where $\gamma > 0$ is a kernel parameter.
    \item \textbf{Sigmoid Kernel:}
    \[
    K(\mathbf{x}_i, \mathbf{x}_j) = \tanh(\gamma \mathbf{x}_i^\top \mathbf{x}_j + r)
    \]
\end{itemize}

\subsection{Dual Formulation}

\begin{itemize}
    \item The SVM optimization problem can be expressed in its dual form, which depends only on the inner products of data points.
    \item Dual problem for soft-margin SVM:
    \[
    \begin{aligned}
    & \underset{\boldsymbol{\alpha}}{\text{maximize}} && \sum_{i=1}^n \alpha_i - \frac{1}{2} \sum_{i=1}^n \sum_{j=1}^n \alpha_i \alpha_j y_i y_j K(\mathbf{x}_i, \mathbf{x}_j) \\
    & \text{subject to} && 0 \leq \alpha_i \leq C, \quad \forall i \\
    &&& \sum_{i=1}^n \alpha_i y_i = 0
    \end{aligned}
    \]
    \item The decision function becomes:
    \[
    f(\mathbf{x}) = \sum_{i=1}^n \alpha_i y_i K(\mathbf{x}_i, \mathbf{x}) + b
    \]
    \item Only support vectors (data points with $\alpha_i > 0$) contribute to the decision function.
\end{itemize}

\subsection{SVM in Practice}

\subsubsection{Pros}

\begin{itemize}
    \item Effective in high-dimensional spaces.
    \item Works well when the number of features exceeds the number of samples.
    \item Uses a subset of training points (support vectors), making it memory efficient.
    \item Flexible with different kernel functions for various data types.
\end{itemize}

\subsubsection{Cons}

\begin{itemize}
    \item Can be computationally intensive for large datasets.
    \item Choosing the right kernel and hyperparameters can be challenging.
    \item SVMs do not provide direct probability estimates (though methods exist to approximate them).
\end{itemize}

\subsubsection{Practical Considerations}

\begin{itemize}
    \item \textbf{Hyperparameter Tuning:}
    \begin{itemize}
        \item Regularization parameter $C$ and kernel parameters (e.g., $\gamma$ in RBF kernel) need to be tuned using methods like cross-validation.
    \end{itemize}
    \item \textbf{Feature Scaling:}
    \begin{itemize}
        \item Feature scaling is important as SVMs are sensitive to the scale of the inputs.
    \end{itemize}
    \item \textbf{Class Imbalance:}
    \begin{itemize}
        \item SVMs may perform poorly when classes are imbalanced; techniques like class weighting can be used.
    \end{itemize}
\end{itemize}

\subsection{Conclusion}

\begin{itemize}
    \item SVMs are powerful tools for classification and regression tasks.
    \item The use of kernel functions allows SVMs to model complex, non-linear relationships.
    \item Proper tuning and consideration of practical aspects are essential for achieving good performance.
\end{itemize}i

\end{document}